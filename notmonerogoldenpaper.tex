\documentclass{article}

\usepackage{hyperref}
\usepackage{graphicx}
\graphicspath{{.images}}

\author{ponymontana \\ \href{mailto:ponymontana@disroot.org}{ponymontana@disroot.org}}
\title{Addendum to an informal pseudoexistent Monero Goldenpaper}
\date{june 14, 2022}\vspace{0.5cm}

\begin{document}

\maketitle

\vspace{0.4cm}
\large{}

\emph{In this article I want to propose a perspective about wealth inequality discriminations in Bitcoin, Monero and gold.} \vspace{0.5cm}

\normalsize{}

In particular I will affirm that Bitcoin has a mechanism of \textbf{vertical wealth discrimination} built in the protocol that:


\begin{enumerate}

        \item make Bitcoin dangerous for fragile people.

	\item make Bitcoin a fraud / make Bitcoiners fraudsters.\vspace{0.3cm}

\end{enumerate}


\section{Wealth inequality by design}


Bitcoin and monero were born to be the exactly same tool: a peer-to-peer, private, uncensurable, electronic currency. Now monero is the best technology to serve the purpose to be that tool; \textbf{bitcoin has failed} to be it. Even bitcoiners agree (at least implicitly) with the fail of bitcoin as currency: bitcoin is now something to stack and conserve.\vspace{0.2cm}

The today main narrative about Bitcoin is that it's "the digital gold", while Monero try to be the anonymous digital cash-like payment system and your anonymous digital bank.\vspace{0.2cm}


These two protocols use the blockchain to achieve the digital scarcity; they also use an emission design that made them effectively scarce on an economic level.


One of the main game-theoretic goal of this design is to create an incentive to buy, mine, or work on the protocol. The early adopters will be rewarded with the price appreciation of the currency when the network grows. The newcoomers pumps the o.g. bags.\vspace{0.2cm}


I call this structure \emph{wealth inequality by design} and I want to know if and when it become fraudulent. \vspace{0.5cm}


\section{When is it fair?}\vspace{0.3cm}

\emph{There are two questions I were trying to answer for long time. This chapter can be intended as an informal representation of the core of this article.}\vspace{0.5cm}

\begin{itemize}
	\item \textbf{When the economic structure of btc/monero is arguably fair} in the grand schema of things.. . and when not?\vspace{0.2cm}


		Answer: \emph{when the last human that entered in the pyramid scheme suffers by protocol the consequences of wealth inequality.. . then is NOT fair.} \vspace{0.2cm}


	\item If I want to use bitcoin and \textbf{I'm the last human who discovered the tool} (and I'm an unbanked lady from afrikan village, the most economically fragile human on earth), \textbf{can I use it}?\vspace{0.2cm}


		Answer: \emph{If today / these days / these months.. . there are low transaction fees, you probably could use it. But the risk to have your wealth trapped on the chain are high. Could arrive a day when you will not be able to use bitcoin. Bitcoin isn't a tool that empowers you. The base fee could become easily superior to your monthly revenue.
		And if some powerful (and they don't even need to be too powerful) people want to abuse and control you (powerful people have the tendency to abuse fragile ones.. .) they can track how you move your btc.}\vspace{0.5cm}

\end{itemize}



\section{Splitting Bitcoin and splitting gold}\vspace{0.4cm}



\textbf{UNIT = 3 * minimun fees to transact}\vspace{0.6cm}




Btc is in theory splittable in 0.00000001 sats, but \emph{that's ridiculous}.

In practice, we could consider it splittable in something like \textbf{UNIT}s of (\textbf{(fees necessary to make a transaction in at least one or two days)} + \textbf{(an amount of btc that's at least 2 times the fees that will be paid to move it)}) = \textbf{(3 times the fees necessary to make a transaction in at least one or two days))}.\vspace{0.2cm}

The UNIT is a variable, subject to speculation, that could be \emph{retrospectively} calculated (\emph{It would be something necessarily higher than the minimun of the minimun fees paid in the last days of blocks and lower than the higher of the average fees}) but it's not intended to be a rigorous math object; it has the only purpose of rapresent the concept of "non-splittable unit" in an intuitive way.\vspace{0.3cm}


\textbf{Splitting Bitcoin in values smaller than UNIT makes impossible any rational use of the tool}.\vspace{0.4cm}


In btc we observe a tendency:


\begin{itemize}

	\item every market cycle the UNIT we described grows (in usd).
	\item as the UNIT grows, also the people that can't use the tool grow in number.\vspace{0.3cm}

\end{itemize}


\subsection{The infinite growing theater}


Bitcoin uses a \textbf{capped 1 mb blocksize} that is \textbf{the bottleneck} when the transactions grow (and the price goes up).\vspace{0.2cm}

It's like using the same small door for a perpetual and infinite growing theater where will enter more and more people daily. It will cause:


\begin{enumerate}


        \item death for fragile people in panic situations like a fire alarm or the end of capitalism;

	\item if the grow tendency will continue, the door will be too small for the daily basis human traffic. It will become too expensive for plebs to go to theater. There will be people that can't enter and, most dramatic and dystopian, people that can't exit. There will be people trapped in the theater!\vspace{0.2cm}


\end{enumerate}



\subsection{Digital gold}


The network wealth inequality in btc is something that empowers rich people (thats ok) and kills -or will kill at some point- the possibility of poor people to use it (thats not ok) by protocol. I'm not discussing the macroeconomic aspect of wealth inequality. I discuss the fact that a lot of humans can't use the tool (even only as a store of value) or will be damaged and abused using it.


\emph{But how can some people call btc "the digital gold"!?}


\textbf{Gold doesn't have this problem and is infinitely splittable}.\vspace{0.5cm}


\section{Same pyramid built on different sands}\vspace{0.3cm}




\subsection{The good: GOLD}


If I buy some gold, I accept that my use of it is de facto a \textbf{legitimation of the power} of people that have a lot of gold.


But the wealth inequality has no effect on the "\emph{basic protocol level of the gold hodl and exchange}"; I can use my little \textbf{1/n kg} of gold in the same way as an oligarch can uses his \textbf{n kg}. This is valid for every positive number, with \emph{n that goes to infinite}; the split limit is atomic.\vspace{0.5cm}




\subsection{The bad: BITCOIN}


If I have \textbf{1/n Btc}, and a bitcoiner have his \textbf{n btc}, we would expect the same situation of gold. But that's valid only for a limited sequence of positive numbers. In fact, need to be \textbf{(my wallet)} bigger than \textbf{UNIT}, as previously explained. For a big \textbf{n}, I loose the possibility to transact while the bitcoiner preserve all his power in the network. And as the UNIT grows (in btc, not in usd by the price appreciation), the n needed to kick me out of the game become smaller.


Today, in june 2022 btc fees are low, I need, approximating, \emph{an amount bigger than 0.00001 btc} to have a wallet that can transact on the network; or we can say \textbf{today btc is splittable in UNITs of 0.00001 btc}.


In april 2021 for example the UNIT reached 0.001 btc. For some days in that month, if you had less than that amount you didn't pratically be able to transact.\vspace{0.2cm}



\textbf{Btc was splittable in UNITs of 0.001 btc; this was an approximated rational-use-limit of Bitcoin}.\vspace{0.2cm}


And btc price was ~55k usd/btc so our unbanked lady from african village needed at least 55 usd value of btc to consider utilize it (with absolutely high and relatively extremely high fees).\vspace{0.3cm}


\emph{If she did have her money in btc at the time (bought in a moment of low fees) she had practically a frozen account}. Even if we consider the pump that her btc could have made from the time she bought, it can't compensate the fees explosion; it's all another order of magnitude. Unless she is an early adopter.\vspace{0.5cm}


\subsection{Possible scenarios}


We can theorize \textbf{two possible situations}:\vspace{0.2cm}

\begin{enumerate}


	\item bitcoin will become \textbf{more and more useful} (at least a bit of adoption) and we will observe:

		\begin{enumerate}
	\item the inevitable UNIT growth in terms of btc and usd (systemic expansion);
	\item the increase of risk and fragility for the poorer on the network (unpredictable non-linear explosion).\vspace{0.6cm}

		\end{enumerate}

	\item bitcoin will become \textbf{useless} (no adoption and rejection on the markets).\vspace{1.2cm}


\end{enumerate}

 In the two scenarios Bitcoin is useless for most people; in the first it is also dangerous.


\subsection{Ok Mr. Saylor, you got me.. .}\vspace{0.3cm}


\emph{Okay, I got it, the key to stop the financial slavery that I live day by day is to buy Bitcoin, okay.}\vspace{0.5cm}


\textbf{I am in the last half of the world} in terms of wealth. I discovered Bitcoin, seems great, the problem solver, my new begin.\vspace{0.2cm}

But Saylor doesn't tell me that if I buy some sats, there will come the days when people richer than me will make impossible for my transactions to enter in the 1 mb block. The probability that I will be abused on the base layer of the protocol \emph{on a systemic base} and/or \emph{in some extraordinary unpredictable (but presumable and ciclical) events} are so high.\vspace{0.3cm}

\textbf{Billionaires that endorsed me bitcoin have literally frauded me with a ponzi scheme that has enriched they, and gave me a useless tool. They literally have stolen my money; I can't even cashout because I can't pay fees!}\vspace{0.2cm}

\textbf{Oh.. .and how damned the police has known that last month I was evading the certanly temporary lockdown buyin some fresh milk in the near village. ..}\vspace{0.5cm}




\subsection{The anon: MONERO}


As I enter in the network buying some xmr and using it as currency and store of value I accept that my use and endorse of monero is a \textbf{legitimation of the wealth power of people that entered in the network first than me} and made some big bags, or the billionaire who will in future buy billions on xmr.\vspace{0.2cm}


But \textbf{how does the protocol discriminate} from me and these richer people? What's the effects of the \emph{wealth inequality by design}?\vspace{0.2cm}



Is Monero more similar to gold where the inequality doesn't affect by any means my use of the tool or is it Bitcoin-like?\vspace{0.2cm}


The UNIT on monero now is really low in usd. The reasons is that there are (relatively) not so much daily transactions.\vspace{0.3cm}


\emph{But there's more}.\vspace{0.3cm}

Monero doesn't have a capped blocksize: it uses well designed algoritmical adjustments to scale while it remain protected from bad actors that want spam the blockchain and do a lot of bad stuff.

\textbf{In monero, by design, as the adoption grow, the UNIT we described earlier will become smaller!} (Hopefully. Algorithms will need to be battle-tested.)\vspace{0.3cm}



As the A.N.O.N. theater grows, also the doors grow.\vspace{0.2cm}


There's an emergency exit/enter door next to the daily-used door in the Monero \emph{-Auspically Not a death trap in case Of Non-predicatble catastrophic events- theater}. \vspace{0.2cm}

If you want to deep dive the math behind that magic, I found technical details explained very well in the chapter 7 of the second edition of "Zero to Monero". Links are at the end of the article!\vspace{0.3cm}

\subsection{Summarazing}


\begin{itemize}

	\item \textbf{GOLD}: UNIT is pratically inexistent - the protocol doesn't discriminate.


	\item \textbf{BITCOIN}: UNIT is not high, will probably be higher, will ciclically be enormous - the protocol discriminate.


	\item \textbf{MONERO}: UNIT is low, will decrease, will hopefully never become so high and not for a long time - we can expect a lot less protocol's-level-discrimination than btc.\vspace{0.2cm}

\end{itemize}


\emph{Okay, from the perspective we have presented, Monero seems to emulate gold better than how btc does; and it is also designed to maintain his characteristics in future}.\vspace{0.2cm}

\textbf{But, is Monero "the digital gold"?}\vspace{0.2cm}


There are a lot of reasons why, \textbf{if a digital gold exists today, than it's Monero} (that also remain the best digital anonimous cash).\vspace{0.4cm}


\textbf{What's mean the strange title of the article?}

\emph{The ideas presented here are intended to be an appendix of the documentation and divulgation work made by the Monero community. The fungibility is the first reason that differentiate Monero from Bitcoin and make Monero gold-like and cash-like. Througt the links at the end you can obtain a lot of info about that.}\vspace{0.2cm}




\section{The future of money}\vspace{0.2cm}



I think that \textbf{it will be an important challange for monero to maintain the accessibility} that bitcoin lost and will inevitably lose.\vspace{0.3cm}



\emph{If we want to make monero the real revolutionary money, the last people who will buy in, and the poorer and most fragile people in the world, need to receive a usable and great tool; the key for financial freedom and independence, for storing and use digital money}.\vspace{0.4cm}

If Monero will fail in this purpose it will become a scam, like Bitcoin is nowadays.



\subsection{Time changes everything}\vspace{0.3cm}


\textbf{Why you need to rethink your positions about Bitcoin}!?\vspace{0.3cm}


\emph{Who were o.g. bitcoiners ten years ago?}\vspace{0.1cm}

\textbf{They were people that understood bitcoin}. They understood the value of open source - uncensorable - private - internet money and how umanity can benefit of it. \emph{They have choose the better tool to serve the purpose of money and store of value. At the time, that tool was bitcoin}.\vspace{0.3cm}


\emph{Who are bitcoiners nowadays?}\vspace{0.1cm}

\textbf{They are people who endorse a tool that is useless and dangerous for the majority of humanity}.

When they tell to a casual person on the internet "man you need btc is fantastic open source money blah blah blah" there's an high probability that \textbf{they endorse a ponzi-scheme} where the person in question will be abused-by-protocol. And the o.g. bitcoiner will be richer.\vspace{0.2cm}

\begin{itemize}

	\item If I insistently convince you to make something that hurts you, then I'm guilty for have caused your damage.
	\item If I insistently convince you to make something that hurts you and enriches me, then I'm guilty for have caused your damage and I'm also a fraudster.\vspace{0.2cm}

\end{itemize}


\subsection{Monerochads, be maximalists!}


\textbf{Bitcoiners were based. Today they are just wrong.} And, unintentionally(?), scammers. And cool people need to take position. The Bitcoin narratives like "is a store of value" or "at least is open source money" need to stop.\vspace{0.5cm}


\textbf{Just be "more bitcoiners than bitcoiners"}.\vspace{0.3cm}

Just do exacly what Bitcoiners did ten years ago: \emph{endorse the best technology that serve the purpose of being money} - TODAY! The peer-to-peer, private, uncensurable, A.N.O.N., electronic currency - TODAY!

Just be cool monero maximalist, don't accept ponzis.\vspace{0.2cm}


And if in future you don't want to be called "scammers" please, \textbf{make sure to keep it a useful tool for everyone}!\vspace{3cm}

That's all. I wrote this article with the intention of stimulate a positive debate in the monero community.\vspace{0.2cm}

This is entirely fruit of voluntary work. I'm not an english native speaker, so it hasn't been so easy (sorry for my bad use of the language, I did my best). \vspace{0.5cm}

\hspace{0.3cm}\textbf{If you find it useful, you can tip me at my monero wallet}:\vspace{0.3cm}


\hspace{-3.5cm} \footnotesize{83MHFVyLnSsahyW4qzQoHXKCytDwrjJTRiFBZHU3ccePY1sPNceMyAEWvjka4nx26RXCr1M6RNhrVBFGagquF693JGYpPVj} \vspace{0.3cm}


\hspace{3cm} \includegraphics[scale=0.02]{monero.png}
\hspace{0.5cm}\includegraphics[scale=0.3]{qrcode.png} \vspace{0.5cm}

\hspace{2cm} \normalsize{} \textbf{Thanks a lot, I really appreciate :)}\vspace{1cm}

\hspace{-0.5cm}If you have any questions don't hesitate to mail me at ponymontana@disroot.org\vspace{4cm}



\textbf{links:}
\footnotesize{}
\begin{itemize}

	\item Historical views of btc and monero average transaction fee in usd:


		\url{https://bitinfocharts.com/comparison/monero-transactionfees.html#log&alltime}


		\url{https://bitinfocharts.com/comparison/bitcoin-transactionfees.html#log&alltime}


\item The books "Zero to Monero" and "mastering monero":



	\url{https://www.getmonero.org/library}



\item A good info point:



	\url{https://moneroinfodump.neocities.org}


\item All the daily news, stories and updates about monero:


	\url{https://monero.observer}

\item Some good video contents and channels:


	\url{https://www.youtube.com/watch?v=qPNxca_KMww}


	\url{https://www.youtube.com/channel/UC3Hx81QYLoEQkm3vyl4N4eQ}


	\url{https://www.youtube.com/watch?v=QrHsFZBab4U}

\end{itemize}

\end{document}
